\documentclass[letterpaper]{article}
\usepackage{fancyhdr}
\pagestyle{fancy}
\lhead{Gregory Croisdale} % controls the left corner of the header
\chead{} % controls the center of the header
\rhead{Intro to Quantum Mechanics S2020 HW1} % controls the right corner of the header
\lfoot{} % controls the left corner of the footer
\cfoot{} % controls the center of the footer
\rfoot{Page~\thepage} % controls the right corner of the footer
\renewcommand{\headrulewidth}{0.1pt}
\renewcommand{\footrulewidth}{0.4pt}
\begin{document}
\begin{section}{Exercise III.1}
  Compute the probability of measuring $|0\rangle$ and $|1\rangle$ for each of the following quantum states:
  \begin{enumerate}
    \item $0.6|0\rangle + 0.8|1\rangle$.
    \begin{itemize}
      \item %YOUR ANSWER HERE%
  \end{itemize}
    \item $\frac{1}{\sqrt{3}}|0\rangle + \sqrt{2/3}|1\rangle$.
    \begin{itemize}
      \item %YOUR ANSWER HERE%
  \end{itemize}
    \item $\frac{\sqrt{3}}{2}|0\rangle - \frac{1}{2}|1\rangle$.
    \begin{itemize}
      \item %YOUR ANSWER HERE%
  \end{itemize}
    \item $-\frac{1}{25}(24|0\rangle - 7|1\rangle)$.
    \begin{itemize}
      \item %YOUR ANSWER HERE%
  \end{itemize}
    \item $-\frac{1}{\sqrt{2}}|0\rangle + \frac{e^i\pi/6}{\sqrt{2}}|1\rangle$.
    \begin{itemize}
      \item %YOUR ANSWER HERE%
  \end{itemize}
  \end{enumerate}
\end{section}
\begin{section}{Exercise III.2}
  Compute the probability of the four states if the following are measured in the computational basis:
  \begin{enumerate}
    \item $(e^i|00\rangle + \sqrt{2}|01\rangle + \sqrt{3}|10\rangle + 2e^2i|11\rangle)/\sqrt{10}$.
    \begin{itemize}
      \item %YOUR ANSWER HERE%
  \end{itemize}
    \item $\frac{1}{2}(-|0\rangle+|1\rangle)\otimes(e^{\pi i}|0\rangle + e ^ {-\pi i}|1\rangle)$.
    \begin{itemize}
      \item %YOUR ANSWER HERE%
  \end{itemize}
    \item $(\sqrt{1/3}|0\rangle - \sqrt{2/3}|1\rangle)\otimes\sqrt{2}(\frac{e^{\pi i/4}}{2}|0\rangle+\frac{e^{\pi i/2}}{2}|1\rangle)$.
    \begin{itemize}
      \item %YOUR ANSWER HERE%
  \end{itemize}
  \end{enumerate}
\end{section}
\begin{section}{Exercise III.3}
  Suppose that a two-quibit register is in the state
  $$|\psi\rangle=\frac{3}{5}|00\rangle - \frac{\sqrt{7}}{5}|01\rangle+\frac{e^{i\pi/2}}{\sqrt{5}}|10\rangle-\frac{2}{5}|11\rangle$$
  \begin{enumerate}
    \item Suppose we measure just the first qubit. Compute the probability of measuring a $|0\rangle$ or a $|1\rangle$ and the resulting register state in each case.
    \begin{itemize}
      \item %YOUR ANSWER HERE%
  \end{itemize}
    \item Do the same, but supposing instead that we measure just the second qubit.
    \begin{itemize}
      \item %YOUR ANSWER HERE%
  \end{itemize}
  \end{enumerate}
\end{section}
\begin{section}{Exercise III.12}
  Show that the four Bell states are orthonormal (i.e., both orthogonal and normalized).
  \begin{itemize}
      \item %YOUR ANSWER HERE%
  \end{itemize}
\end{section}
\begin{section}{Exercise III.13}
  Prove that $|\beta_{11}\rangle$ is entangled.
  \begin{itemize}
      \item %YOUR ANSWER HERE%
  \end{itemize}
\end{section}
\begin{section}{Exercise III.14}
  Prove that $\frac{1}{\sqrt{2}}(|000\rangle + |111\rangle)$ is entangled.
  \begin{itemize}
      \item %YOUR ANSWER HERE%
  \end{itemize}
\end{section}
\begin{section}{Exercise III.30}
  Use a single Toffoli gate to implement each of NOT, NAND, and XOR.
  \begin{itemize}
      \item %YOUR ANSWER HERE%
  \end{itemize}
\end{section}
\begin{section}{Exercise III.34}
  Prove:
  $$|0\rangle=\frac{1}{\sqrt{2}}(|+\rangle+|-\rangle),$$
  \begin{itemize}
      \item %YOUR ANSWER HERE%
  \end{itemize}
  $$|1\rangle=\frac{1}{\sqrt{2}}(|+\rangle-|-\rangle).$$
  \begin{itemize}
      \item %YOUR ANSWER HERE%
  \end{itemize}
\end{section}
\begin{section}{Exercise III.35}
  What are the possible outcomes (probabilities and resulting states) of measuring $a|+\rangle+b|-\rangle$ in the \emph{computational basis} (of course, $|a|^2+|b|^2=1$)?
  \begin{itemize}
      \item %YOUR ANSWER HERE%
  \end{itemize}
\end{section}
\end{document}
